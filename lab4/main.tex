%%%%%%%%%%%%%%%%%%%%%%%%%%%%%%%%%%%%%%%%%
% Wenneker Assignment
% LaTeX Template
% Version 2.0 (12/1/2019)
%
% This template originates from:
% http://www.LaTeXTemplates.com
%
% Authors:
% Vel (vel@LaTeXTemplates.com)
% Frits Wenneker
%
% License:
% CC BY-NC-SA 3.0 (http://creativecommons.org/licenses/by-nc-sa/3.0/)
%
%%%%%%%%%%%%%%%%%%%%%%%%%%%%%%%%%%%%%%%%%

%----------------------------------------------------------------------------------------
%	PACKAGES AND OTHER DOCUMENT CONFIGURATIONS
%----------------------------------------------------------------------------------------

\documentclass[11pt]{scrartcl} % Font size

%%%%%%%%%%%%%%%%%%%%%%%%%%%%%%%%%%%%%%%%%
% Wenneker Assignment
% Structure Specification File
% Version 2.0 (12/1/2019)
%
% This template originates from:
% http://www.LaTeXTemplates.com
%
% Authors:
% Vel (vel@LaTeXTemplates.com)
% Frits Wenneker
%
% License:
% CC BY-NC-SA 3.0 (http://creativecommons.org/licenses/by-nc-sa/3.0/)
%
%%%%%%%%%%%%%%%%%%%%%%%%%%%%%%%%%%%%%%%%%

%----------------------------------------------------------------------------------------
%	PACKAGES AND OTHER DOCUMENT CONFIGURATIONS
%----------------------------------------------------------------------------------------

\usepackage{amsmath, amsfonts, amsthm} % Math packages

\usepackage{listings} % Code listings, with syntax highlighting

\usepackage[english]{babel} % English language hyphenation

\usepackage[xetex]{graphicx}
%\usepackage{graphicx} % Required for inserting images
\graphicspath{{Figures/}{./}} % Specifies where to look for included images (trailing slash required)

\usepackage{booktabs} % Required for better horizontal rules in tables

\numberwithin{equation}{section} % Number equations within sections (i.e. 1.1, 1.2, 2.1, 2.2 instead of 1, 2, 3, 4)
\numberwithin{figure}{section} % Number figures within sections (i.e. 1.1, 1.2, 2.1, 2.2 instead of 1, 2, 3, 4)
\numberwithin{table}{section} % Number tables within sections (i.e. 1.1, 1.2, 2.1, 2.2 instead of 1, 2, 3, 4)

\setlength\parindent{0pt} % Removes all indentation from paragraphs

\usepackage{enumitem} % Required for list customisation
\setlist{noitemsep} % No spacing between list items

%----------------------------------------------------------------------------------------
%	DOCUMENT MARGINS
%----------------------------------------------------------------------------------------

\usepackage{geometry} % Required for adjusting page dimensions and margins

\geometry{
	paper=a4paper, % Paper size, change to letterpaper for US letter size
	top=2.5cm, % Top margin
	bottom=3cm, % Bottom margin
	left=3cm, % Left margin
	right=3cm, % Right margin
	headheight=0.75cm, % Header height
	footskip=1.5cm, % Space from the bottom margin to the baseline of the footer
	headsep=0.75cm, % Space from the top margin to the baseline of the header
	%showframe, % Uncomment to show how the type block is set on the page
}

%----------------------------------------------------------------------------------------
%	FONTS
%----------------------------------------------------------------------------------------

\usepackage[utf8]{inputenc} % Required for inputting international characters
\usepackage[T1]{fontenc} % Use 8-bit encoding

\usepackage{fourier} % Use the Adobe Utopia font for the document

%----------------------------------------------------------------------------------------
%	SECTION TITLES
%----------------------------------------------------------------------------------------

\usepackage{sectsty} % Allows customising section commands

\sectionfont{\vspace{6pt}\centering\normalfont\scshape} % \section{} styling
\subsectionfont{\normalfont\bfseries} % \subsection{} styling
\subsubsectionfont{\normalfont\itshape} % \subsubsection{} styling
\paragraphfont{\normalfont\scshape} % \paragraph{} styling

%----------------------------------------------------------------------------------------
%	HEADERS AND FOOTERS
%----------------------------------------------------------------------------------------

\usepackage{scrlayer-scrpage} % Required for customising headers and footers

\ohead*{} % Right header
\ihead*{} % Left header
\chead*{} % Centre header

\ofoot*{} % Right footer
\ifoot*{} % Left footer
\cfoot*{\pagemark} % Centre footer
 % Include the file specifying the document structure and custom commands
% LaTeX settings for MATLAB code listings
% based on Ted Pavlic's settings in http://links.tedpavlic.com/ascii/homework_new_tex.ascii
\usepackage{listings}
\usepackage[usenames,dvipsnames]{color}

% This is the color used for MATLAB comments below
\definecolor{MyDarkGreen}{rgb}{0.0,0.4,0.0}

% For faster processing, load Matlab syntax for listings
\lstloadlanguages{Matlab}%
\lstset{language=Matlab,                        % Use MATLAB
        frame=single,                           % Single frame around code
        basicstyle=\scriptsize\ttfamily,             % Use small true type font
        keywordstyle=[1]\color{Blue}\bfseries,        % MATLAB functions bold and blue
        keywordstyle=[2]\color{Purple},         % MATLAB function arguments purple
        keywordstyle=[3]\color{Blue}\underbar,  % User functions underlined and blue
        identifierstyle=,                       % Nothing special about identifiers
                                                % Comments small dark green courier
        commentstyle=\usefont{T1}{pcr}{m}{sl}\color{MyDarkGreen}\small,
        stringstyle=\color{Purple},             % Strings are purple
        showstringspaces=false,                 % Don't put marks in string spaces
        tabsize=3,                              % 5 spaces per tab
        %
        %%% Put standard MATLAB functions not included in the default
        %%% language here
        morekeywords={xlim,ylim,var,alpha,factorial,poissrnd,normpdf,normcdf,imresize,double,immse,fspecial,cell2mat,circshift,cell},
        %
        %%% Put MATLAB function parameters here
        morekeywords=[2]{on, off, interp},
        %
        %%% Put user defined functions here
        morekeywords=[3]{FindESS, homework_example},
        %
        morecomment=[l][\color{Blue}]{...},     % Line continuation (...) like blue comment
        numbers=left,                           % Line numbers on left
        firstnumber=1,                          % Line numbers start with line 1
        numberstyle=\tiny\color{Blue},          % Line numbers are blue
        stepnumber=1                            % Line numbers go in steps of 5
        }

% Includes a MATLAB script.
% The first parameter is the label, which also is the name of the script
%   without the .m.
% The second parameter is the optional caption.
\newcommand{\matlabscript}[2]
  {\begin{itemize}\item[]\lstinputlisting[caption=#2,label=#1]{#1.m}\end{itemize}}


\usepackage{fontspec}
\setmainfont{Tinos Nerd Font} %nice font for english and greek

\usepackage{hyperref} %for hyperlinks
\hypersetup{
    colorlinks=true,
    linkcolor=blue,
    filecolor=magenta,
    urlcolor=cyan,
}
%----------------------------------------------------------------------------------------
%	TITLE SECTION
%----------------------------------------------------------------------------------------

\title{
	\normalfont\normalsize
	\textsc{Technical University of Crete, ECE}\\ % Your university, school and/or department name(s)
	\vspace{25pt} % Whitespace
	\rule{\linewidth}{0.5pt}\\ % Thin top horizontal rule
	\vspace{20pt} % Whitespace
	{\Huge Digital Image Processing}\\ % The assignment title

	{\huge Fourth Lab Report}\\ % The assignment title
	\vspace{12pt} % Whitespace
	\rule{\linewidth}{2pt}\\ % Thick bottom horizontal rule
	\vspace{12pt} % Whitespace
}

\author{\LARGE{Τσιαούσης Χρήστος}\\
		\texttt{2016030017}
		\and
		\LARGE{Πρωτοπαπαδάκης Γιώργος}\\
		\texttt{2016030134}}% Your name

\date{\normalsize\today} % Today's date (\today) or a custom date

\begin{document}

\maketitle % Print the title

%----------------------------------------------------------------------------------------
%	FIGURE EXAMPLE
%----------------------------------------------------------------------------------------

\section{Σκοπός Εργαστηρίου}

Το εργαστήριο έχει ως σκοπό την περεταίρω εξοικείωση μας με την συνέλιξη σε δισδιάστατα σήματα όπως μια grayscale εικόνα καθώς και την κατανόηση του
θορύβου και πως μπορούμε να τον μειώσουμε. Δηλαδή φίλτράρισμα για μείωση θορύβου και για εμφάνιση πληροφορίας.
%------------------------------------------------

\section{Excersize 1}

Για την κατασκευή της Compute Median, αξιοποιήσαμε την συνάρτηση της συνέλιξης, που είχαμε υλοποιήσει στα προηγούμενα εργαστήρια, καθώς και την συνάρτηση
των paddings. Η μόνη παραλλαγή είναι οτι αντί να αθροίσουμε τα επι μέρους κελιά, τα μετατρέπουμε σε ένα μονοδιάστατο πίνακα και παίρνουμε την μεσαία τιμή.

Το αποτέλεσμα είναι να αγνοείται περισσότερος θόρυβος, όσο περισσότερο αυξάνεται το μέγεθος του φίλτρου.


\begin{figure}[h]
    \centering
    \makebox[\textwidth]{\includegraphics[width=0.7\paperwidth]{3x3.jpg}}
    \caption{Οι εικόνες με φίλτρο 3x3.}
\end{figure}
\begin{figure}[h]
    \centering
    \makebox[\textwidth]{\includegraphics[width=0.7\paperwidth]{5x5.jpg}}
    \caption{Οι εικόνες με φίλτρο 5x5.}
\end{figure}
\begin{figure}[h]
    \centering
    \makebox[\textwidth]{\includegraphics[width=0.7\paperwidth]{9x9.jpg}}
    \caption{Οι εικόνες με φίλτρο 9x9.}
\end{figure}


\clearpage
\subsection*{Κώδικας συνάρτηση Compute Median}
\matlabscript {Compute_Median}{Η συνάρτηση Compute Median.}

\clearpage
\section{Excersize 2}

Η διαδικασία είναι όμοια με αυτήν της πρώτης άσκησης, με την διαφορά πως στην περίπτωση του Minimum, εισάγουμε το πρώτο κελί του sorted 1D-array
στον πίνακα επιστροφής, ενώ για την Maximum το τελευταίο. Βλέπουμε ότι καθώς αυξάνουμε το μέγεθος του φίλτρου, αυξάνεται και η πιθανότητα για ακραίες
τιμές, με αποτέλεσμα στην περίπτωση του \textit{9x9}, το computeMin να δείνει σχεδόν καθολικά μαύρο (τιμή 0 στο colorbar), και αντίστοιχα
το computeMax σχεδόν παντού 255 (άσπρο).

\begin{figure}[h]
    \centering
    \makebox[\textwidth]{\includegraphics[width=0.7\paperwidth]{3x3_1.jpg}}
    \caption{Η πρώτη εικόνα με φίλτρο 3x3.}
\end{figure}
\begin{figure}[h]
    \centering
    \makebox[\textwidth]{\includegraphics[width=0.7\paperwidth]{3x3_2.jpg}}
    \caption{Η δεύτερη εικόνα με φίλτρο 3x3.}
\end{figure}


\begin{figure}[h]
    \centering
    \makebox[\textwidth]{\includegraphics[width=0.7\paperwidth]{5x5_1.jpg}}
    \caption{Η πρώτη εικόνα με φίλτρο 5x5.}
\end{figure}
\begin{figure}[h]
    \centering
    \makebox[\textwidth]{\includegraphics[width=0.7\paperwidth]{5x5_2.jpg}}
    \caption{Η δεύτερη εικόνα με φίλτρο 5x5.}
\end{figure}

\begin{figure}[h]
    \centering
    \makebox[\textwidth]{\includegraphics[width=0.7\paperwidth]{9x9_1.jpg}}
    \caption{Η πρώτη εικόνα με φίλτρο 9x9.}
\end{figure}

\begin{figure}[h]
    \centering
    \makebox[\textwidth]{\includegraphics[width=0.7\paperwidth]{9x9_2.jpg}}
    \caption{Η δεύτερη εικόνα με φίλτρο 9x9.}
\end{figure}

\clearpage
\subsection*{Κώδικες συναρτήσεων Compute Mix/Max}

\matlabscript {Compute_Min}{Η συνάρτηση Compute Minimum.}
\matlabscript {Compute_Max}{Η συνάρτηση Compute Maximum.}

\clearpage
\section{Bonus Excersize}

Τέλος, υλοποιήσαμε ένα edge-detecting φίλτρο βάσει του \[ F = [-1 0 1]\] το οποίο μένει ως έχει και συνελίσεται μια φορά με την εικόνα,
κι έπειτα κάνουμε περιστροφή 90$^{\circ}$ και συνελίσουμε ακόμα μία φορά. Το αποτέλεσμα είναι το παρακάτω

\begin{figure}[h]
    \centering
    \makebox[\textwidth]{\includegraphics[width=0.7\paperwidth]{bonus.jpg}}
    \caption{Η εικόνα της έξτρα άσκησης.}
\end{figure}

\clearpage
\subsection*{Κώδικας συνάρτησης Bonus}

\matlabscript {convolution_bonus}{Η συνάρτηση που συνελίσει με μη-τετραγωνικό μονοδιάστατο φίλτρο.}

% \newpage

\section{Συμπεράσματα}
Οι δυνατότητες της ψηφιακής επεξεργασίας εικόνας με παραλλαγές της συνέλιξης, διαφορετικών φίλτρων, και διαφορετικής σειράς
εφαρμογής, είναι άπειρες. Μπορούμε να ενισχύσουμε κατ επιλογίν όποιο κομμάτι πληροφορίας επιλέξουμε. Μπορούμε να ενισχύσουμε
την χρήσιμη πληροφορία, τον θόρυβο ή τις ακμές.


\section{Ο κώδικας που υλοποίησε τα παραπάνω}


\matlabscript {main_4}{Main function for excersize 1.}

\matlabscript {main_4_2}{Main function for excersize 2.}

\matlabscript {bonus_main}{Main function for bonus excersize.}

\matlabscript {padForConv}{Η υλοποίησή μας για τα paddings}

\matlabscript {convolution}{Η υλοποίησή μας για την συνέλιξη}

\end{document}
