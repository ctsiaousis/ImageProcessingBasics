%%%%%%%%%%%%%%%%%%%%%%%%%%%%%%%%%%%%%%%%%
% Wenneker Assignment
% LaTeX Template
% Version 2.0 (12/1/2019)
%
% This template originates from:
% http://www.LaTeXTemplates.com
%
% Authors:
% Vel (vel@LaTeXTemplates.com)
% Frits Wenneker
%
% License:
% CC BY-NC-SA 3.0 (http://creativecommons.org/licenses/by-nc-sa/3.0/)
%
%%%%%%%%%%%%%%%%%%%%%%%%%%%%%%%%%%%%%%%%%

%----------------------------------------------------------------------------------------
%	PACKAGES AND OTHER DOCUMENT CONFIGURATIONS
%----------------------------------------------------------------------------------------

\documentclass[11pt]{scrartcl} % Font size

\input{structure.tex} % Include the file specifying the document structure and custom commands
\input{matlab.tex}

\usepackage{fontspec}
\setmainfont{Tinos Nerd Font} %nice font for english and greek

\usepackage{hyperref} %for hyperlinks
\hypersetup{
    colorlinks=true,
    linkcolor=blue,
    filecolor=magenta,
    urlcolor=cyan,
}
%----------------------------------------------------------------------------------------
%	TITLE SECTION
%----------------------------------------------------------------------------------------

\title{
	\normalfont\normalsize
	\textsc{Technical University of Crete, ECE}\\ % Your university, school and/or department name(s)
	\vspace{25pt} % Whitespace
	\rule{\linewidth}{0.5pt}\\ % Thin top horizontal rule
	\vspace{20pt} % Whitespace
	{\Huge Digital Image Processing}\\ % The assignment title

	{\huge Fourth Lab Report}\\ % The assignment title
	\vspace{12pt} % Whitespace
	\rule{\linewidth}{2pt}\\ % Thick bottom horizontal rule
	\vspace{12pt} % Whitespace
}

\author{\LARGE{Τσιαούσης Χρήστος}\\
		\texttt{2016030017}
		\and
		\LARGE{Πρωτοπαπαδάκης Γιώργος}\\
		\texttt{2016030134}}% Your name

\date{\normalsize\today} % Today's date (\today) or a custom date

\begin{document}

\maketitle % Print the title

%----------------------------------------------------------------------------------------
%	FIGURE EXAMPLE
%----------------------------------------------------------------------------------------

\section{Σκοπός Εργαστηρίου}

Το εργαστήριο έχει ως σκοπό την περεταίρω εξοικείωση μας με την συνέλιξη σε δισδιάστατα σήματα όπως μια grayscale εικόνα καθώς και την κατανόηση του
θορύβου και πως μπορούμε να τον μειώσουμε. Δηλαδή φίλτράρισμα για μείωση θορύβου και για εμφάνιση πληροφορίας.
%------------------------------------------------

\section{Excersize 1}

Για την κατασκευή της Compute Median, αξιοποιήσαμε την συνάρτηση της συνέλιξης, που είχαμε υλοποιήσει στα προηγούμενα εργαστήρια, καθώς και την συνάρτηση
των paddings. Η μόνη παραλλαγή είναι οτι αντί να αθροίσουμε τα επι μέρους κελιά, τα μετατρέπουμε σε ένα μονοδιάστατο πίνακα και παίρνουμε την μεσαία τιμή.

Το αποτέλεσμα είναι να αγνοείται περισσότερος θόρυβος, όσο περισσότερο αυξάνεται το μέγεθος του φίλτρου.


\begin{figure}[h]
    \centering
    \makebox[\textwidth]{\includegraphics[width=0.7\paperwidth]{3x3.jpg}}
    \caption{Οι εικόνες με φίλτρο 3x3.}
\end{figure}
\begin{figure}[h]
    \centering
    \makebox[\textwidth]{\includegraphics[width=0.7\paperwidth]{5x5.jpg}}
    \caption{Οι εικόνες με φίλτρο 5x5.}
\end{figure}
\begin{figure}[h]
    \centering
    \makebox[\textwidth]{\includegraphics[width=0.7\paperwidth]{9x9.jpg}}
    \caption{Οι εικόνες με φίλτρο 9x9.}
\end{figure}


\clearpage
\subsection*{Κώδικας συνάρτηση Compute Median}
\matlabscript {Compute_Median}{Η συνάρτηση Compute Median.}

\clearpage
\section{Excersize 2}

Η διαδικασία είναι όμοια με αυτήν της πρώτης άσκησης, με την διαφορά πως στην περίπτωση του Minimum, εισάγουμε το πρώτο κελί του sorted 1D-array
στον πίνακα επιστροφής, ενώ για την Maximum το τελευταίο. Βλέπουμε ότι καθώς αυξάνουμε το μέγεθος του φίλτρου, αυξάνεται και η πιθανότητα για ακραίες
τιμές, με αποτέλεσμα στην περίπτωση του \textit{9x9}, το computeMin να δείνει σχεδόν καθολικά μαύρο (τιμή 0 στο colorbar), και αντίστοιχα
το computeMax σχεδόν παντού 255 (άσπρο).

\begin{figure}[h]
    \centering
    \makebox[\textwidth]{\includegraphics[width=0.7\paperwidth]{3x3_1.jpg}}
    \caption{Η πρώτη εικόνα με φίλτρο 3x3.}
\end{figure}
\begin{figure}[h]
    \centering
    \makebox[\textwidth]{\includegraphics[width=0.7\paperwidth]{3x3_2.jpg}}
    \caption{Η δεύτερη εικόνα με φίλτρο 3x3.}
\end{figure}


\begin{figure}[h]
    \centering
    \makebox[\textwidth]{\includegraphics[width=0.7\paperwidth]{5x5_1.jpg}}
    \caption{Η πρώτη εικόνα με φίλτρο 5x5.}
\end{figure}
\begin{figure}[h]
    \centering
    \makebox[\textwidth]{\includegraphics[width=0.7\paperwidth]{5x5_2.jpg}}
    \caption{Η δεύτερη εικόνα με φίλτρο 5x5.}
\end{figure}

\begin{figure}[h]
    \centering
    \makebox[\textwidth]{\includegraphics[width=0.7\paperwidth]{9x9_1.jpg}}
    \caption{Η πρώτη εικόνα με φίλτρο 9x9.}
\end{figure}

\begin{figure}[h]
    \centering
    \makebox[\textwidth]{\includegraphics[width=0.7\paperwidth]{9x9_2.jpg}}
    \caption{Η δεύτερη εικόνα με φίλτρο 9x9.}
\end{figure}

\clearpage
\subsection*{Κώδικες συναρτήσεων Compute Mix/Max}

\matlabscript {Compute_Min}{Η συνάρτηση Compute Minimum.}
\matlabscript {Compute_Max}{Η συνάρτηση Compute Maximum.}

\clearpage
\section{Bonus Excersize}

Τέλος, υλοποιήσαμε ένα edge-detecting φίλτρο βάσει του \[ F = [-1 0 1]\] το οποίο μένει ως έχει και συνελίσεται μια φορά με την εικόνα,
κι έπειτα κάνουμε περιστροφή 90$^{\circ}$ και συνελίσουμε ακόμα μία φορά. Το αποτέλεσμα είναι το παρακάτω

\begin{figure}[h]
    \centering
    \makebox[\textwidth]{\includegraphics[width=0.7\paperwidth]{bonus.jpg}}
    \caption{Η εικόνα της έξτρα άσκησης.}
\end{figure}

\clearpage
\subsection*{Κώδικας συνάρτησης Bonus}

\matlabscript {convolution_bonus}{Η συνάρτηση που συνελίσει με μη-τετραγωνικό μονοδιάστατο φίλτρο.}

% \newpage

\section{Συμπεράσματα}
Οι δυνατότητες της ψηφιακής επεξεργασίας εικόνας με παραλλαγές της συνέλιξης, διαφορετικών φίλτρων, και διαφορετικής σειράς
εφαρμογής, είναι άπειρες. Μπορούμε να ενισχύσουμε κατ επιλογίν όποιο κομμάτι πληροφορίας επιλέξουμε. Μπορούμε να ενισχύσουμε
την χρήσιμη πληροφορία, τον θόρυβο ή τις ακμές.


\section{Ο κώδικας που υλοποίησε τα παραπάνω}


\matlabscript {main_4}{Main function for excersize 1.}

\matlabscript {main_4_2}{Main function for excersize 2.}

\matlabscript {bonus_main}{Main function for bonus excersize.}

\matlabscript {padForConv}{Η υλοποίησή μας για τα paddings}

\matlabscript {convolution}{Η υλοποίησή μας για την συνέλιξη}

\end{document}
