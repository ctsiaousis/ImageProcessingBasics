\documentclass[11pt]{scrartcl} % Font size

\input{structure.tex} % Include the file specifying the document structure and custom commands
\input{matlab.tex}

\usepackage{fontspec}
\setmainfont{Tinos Nerd Font} %nice font for english and greek

\usepackage{hyperref} %for hyperlinks
\hypersetup{
    colorlinks=true,
    linkcolor=blue,
    filecolor=magenta,
    urlcolor=cyan,
}
%----------------------------------------------------------------------------------------
%	TITLE SECTION
%----------------------------------------------------------------------------------------

\title{
	\normalfont\normalsize
	\textsc{Technical University of Crete, ECE}\\ % Your university, school and/or department name(s)
	\vspace{25pt} % Whitespace
	\rule{\linewidth}{0.5pt}\\ % Thin top horizontal rule
	\vspace{20pt} % Whitespace
	{\Huge Digital Image Processing}\\ % The assignment title

	{\huge Fifth Lab Report}\\ % The assignment title
	\vspace{12pt} % Whitespace
	\rule{\linewidth}{2pt}\\ % Thick bottom horizontal rule
	\vspace{12pt} % Whitespace
}

\author{\LARGE{Τσιαούσης Χρήστος}\\
		\texttt{2016030017}
		\and
		\LARGE{Πρωτοπαπαδάκης Γιώργος}\\
		\texttt{2016030134}}% Your name

\date{\normalsize\today} % Today's date (\today) or a custom date

\begin{document}

\maketitle % Print the title

\section{Σκοπός Εργαστηρίου}

\begin{figure}[h]
    \centering
    \makebox[\textwidth]{\includegraphics[width=0.4\paperwidth]{cell.png}}
    \caption{Original Image.}
\end{figure}

Το εργαστήριο έχει ως σκοπό την εξοικείωση μας με τις έννοιες μορφολογικής επεξεργασίας εικόνας. Πιο συγκεκριμένα καλούμαστε
να αναγνωρίσουμε το περίγραμμα του παραπάνω κυττάρου με ακρίβεια. Αυτό θα το επιχειρήσουμε χρησιμοποιώντας την μέθοδο Otsu
καθώς και μορφολογικές τεχνικές όπως erode και dilate συνδιασμένες με ποικίλα structuring elements.

\section{Ανάλυση}

\subsection{Max - Min}
Πριν εφαρμόσουμε την μέθοδο του Otsu, έπρεπε να γίνει ένα edge detection στην φωτογραφία, καθώς αν δεν το κάναμε θα είχαμε
μόνο την φωτεινή μεριά του περιγράμματος. Μετά απο αρκετό πειραματισμό, καθώς δεν θελαμε να χρησιμοποιήσουμε την μέθοδο
\textit{edge()} του matlab, καταλήξαμε να χρησιμοποιούμε τεχνικές από το προηγούμενο εργαστήριο. Πιο συγκεκριμένα, εξάγαμε
τις ελάχιστες τιμές της εικόνας και τις αφαιρέσαμε από τις μέγιστες χρησιμοποιώντας έναν 5x5 kernel. Αυτό μας έδωσε ένα
ομοιόμορφο περίγραμμα από το οποίο και μείναμε ικανοποιημένοι. Κάτι αντίστοιχο θα μπορούσε ίσως να επιτευχθεί χρησιμοποιόντας
και την αντίστοιχη διαφορά από morphological opening και closing. Στο παρακάτω figure, μπορεί κανείς να δει τα αποτελέσματά μας.

\begin{figure}[h]
    \centering
    \makebox[\textwidth]{\includegraphics[width=\paperwidth]{1.jpg}}
    \caption{Our Edge Detection Approach.}
\end{figure}
\clearpage

\subsection{EdgeDetection and Otsu}
Στην συνέχεια, χρησιμοποιήσαμε πάνω στο παραπάνω αποτέλεσμα την υλοποίησή μας από το προηγούμενο εργαστήριο για
edge-detection χρησιμοποιώντας το φίλτρο F=[-1 0 1] και την συνάρτηση μας \textit{convolution\_bonus()}. Τέλος,
χρησιμοποιήσαμε στο αποτέλεσμα αυτό τις συναρτήσεις \textit{graythresh()} και \textit{im2bw()} προκειμένου
να εξάγουμε ένα κατώφλι και να κάνουμε την εικόνα binary.

\begin{figure}[h]
    \centering
    \makebox[\textwidth]{\includegraphics[width=\paperwidth]{2.jpg}}
    \caption{Otsu Method Result.}
\end{figure}
\clearpage

\subsection{Contour Connection}
Λόγω τις προηγούμενης επεξεργασίας, έχουμε ήδη ένα αρκετά καλό αποτέλεσμα και θα μπορούσαμε να συνεχίσουμε έτσι στο επόμενο
βήμα. Παρ` όλ' αυτά, επιλέξαμε να χρησιμοποιήσουμε ένα structuring element δίσκου ακτίνας 12 ως όρισμα στη συνάρτηση
\textit{imtophat()} και πήραμε ένα μικρό, αλλά αποτελεσματικό smoothing στο περίγραμμα. έπειτα χρησιμοποιήσαμε την
\textit{imfill()} με το όρισμα `holes' και έτσι είχαμε πλέον αναγνωρήσει όλη την περιοχή του κυττάρου. Το αποτέλεσμα
φαίνεται παρακάτω.

\begin{figure}[h]
    \centering
    \makebox[\textwidth]{\includegraphics[width=\paperwidth]{3.jpg}}
    \caption{Dilate and fill.}
\end{figure}
\clearpage

\subsection{Erode}
Μετα, έπρεπε να μικρύνουμε ελαφώς το περίγραμμα καθώς λόγω της μεθοδολογίας μας ήταν ελαφρώς διογκωμένο.
Επιλέξαμε να χρησιμοποιήσουμε δύο φορές την \textit{imerode('diamond',1)}, χρησιμοποιήθηκε το συγκεκριμένο
structuring element επειδή θέλαμε να εφαρμοστεί κυρίως στις έντονες καμπύλες του περιγράμματος.

Και τέλος αφαιρέσαμε το κύταρρο που δεν θέλαμε να αναγνωρίσουμε μέσω της συνάρτησης \textit{imerode}.

Παρακάτω φαίνονται τα τελικά μας αποτελέσματα.

\begin{figure}[h]
    \centering
    \makebox[\textwidth]{\includegraphics[width=\paperwidth]{4.jpg}}
    \caption{Eroded, cleared-boarder and results.}
\end{figure}
\clearpage

\clearpage
\section{Κώδικας}
\matlabscript {main_5}{Η main.}
\matlabscript {convolution_bonus}{Η συνάρτηση edge-detection.}
\matlabscript {Compute_Min}{Η συνάρτηση Compute Min.}
\matlabscript {Compute_Max}{Η συνάρτηση Compute Max.}

\end{document}
