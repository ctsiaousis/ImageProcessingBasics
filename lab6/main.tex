\documentclass[11pt]{scrartcl} % Font size

\input{structure.tex} % Include the file specifying the document structure and custom commands
\input{matlab.tex}

\usepackage{fontspec}
\setmainfont{Tinos Nerd Font} %nice font for english and greek

\usepackage{hyperref} %for hyperlinks
\hypersetup{
    colorlinks=true,
    linkcolor=blue,
    filecolor=magenta,
    urlcolor=cyan,
}
%----------------------------------------------------------------------------------------
%	TITLE SECTION
%----------------------------------------------------------------------------------------

\title{
	\normalfont\normalsize
	\textsc{Technical University of Crete, ECE}\\ % Your university, school and/or department name(s)
	\vspace{25pt} % Whitespace
	\rule{\linewidth}{0.5pt}\\ % Thin top horizontal rule
	\vspace{20pt} % Whitespace
	{\Huge Digital Image Processing}\\ % The assignment title

	{\huge Sixth Lab Report}\\ % The assignment title
	\vspace{12pt} % Whitespace
	\rule{\linewidth}{2pt}\\ % Thick bottom horizontal rule
	\vspace{12pt} % Whitespace
}

\author{\LARGE{Τσιαούσης Χρήστος}\\
		\texttt{2016030017}
		\and
		\LARGE{Πρωτοπαπαδάκης Γιώργος}\\
		\texttt{2016030134}}% Your name

\date{\normalsize\today} % Today's date (\today) or a custom date

\begin{document}

\maketitle % Print the title

\section{Σκοπός Εργαστηρίου}

\begin{figure}[h]
    \centering
    \makebox[\textwidth]{\includegraphics[width=0.4\paperwidth]{village.jpg}}
    \caption{Original Image.}
\end{figure}

Το εργαστήριο έχει ως σκοπό την περεταίρω εξοικείωση μας με τις έννοιες μορφολογικής επεξεργασίας εικόνας. Πιο συγκεκριμένα καλούμαστε
να αναγνωρίσουμε μία κατοικημένη περιοχή.

\section{Βήμα 1}
Επιτυχημένη αναγνώρηση θεωρήσαμε την εμφάνηση κυρίως του περιγράμματος της κατοικημένης περιοχής αλλά και των διάσπαρτων σπιτιών
που υπάρχουν στα χωράφια. Το καλύτερο αποτέλεσμα για να φαίνονται και τα δύο το βρήκαμε με το κατώφλι να είναι $0.21345$

\begin{figure}[h]
    \centering
    \makebox[\textwidth]{\includegraphics[width=0.6\paperwidth]{1.jpg}}
    \caption{Urban detection using binarization.}
\end{figure}
\clearpage

\section{UrbanDetec}
Στην συνέχεια, χρησιμοποιήσαμε την συνάρτηση που μας δίνεται, με παραμέτρους για παράθυρο 3 και για κατώφλι το ίδιο με πριν.

\begin{figure}[h]
    \centering
    \makebox[\textwidth]{\includegraphics[width=0.6\paperwidth]{2.jpg}}
    \caption{UrbanDetec Result.}
\end{figure}
\begin{figure}[h]
    \centering
    \makebox[\textwidth]{\includegraphics[width=0.6\paperwidth]{3.jpg}}
    \caption{UrbanDetec Result.}
\end{figure}

Σε αυτή τη μέθοδο βλέπουμε καλύτερη διαχείριση της πληροφορίας, καθώς η ανάλυση και η επεξεργασία της εικόνας γίνεται σε ένα μικρό παράθυρο
κάθε φορά. Κι έτσι, σε περιοχές όπου για παράδειγμα το γρασίδι είναι πολύ φωτεινό, αντί να κατηγοριοποιηθεί ως πάνω απο το κατώφλι και να
πάρουμε όλη την περιοχή, γίνεται πιο αυστηρή κατηγοριοποίηση, και οι πιθανότητες να πάρουμε σωστή αναγνώρηση ενός κτηρίου σε ένα φωτεινό χωράφι, είναι
μεγαλύτερες.
\clearpage

\section{Βήμα 3}

Αρχικά υπολογίζουμε τις TOP-Hat και BOT-Hat εικόνες, και επιβεβαιώνουμε το αποτέλεσμα μας μέσω των έτοιμων συναρτήσεων.

\begin{figure}[h]
    \centering
    \makebox[\textwidth]{\includegraphics[width=0.6\paperwidth]{4.jpg}}
    \caption{TOP-Hat and BOT-Hat Images.}
\end{figure}

\clearpage
Έπειτα τις κανονικοποιούμε μέσω της συνάρτησης \textit{im2double()}. Και κάνουμε binarize βάσει της τεχνικής Otsu.

\begin{figure}[h]
    \centering
    \makebox[\textwidth]{\includegraphics[width=0.6\paperwidth]{5.jpg}}
    \caption{bin-TOP-Hat and bin-BOT-Hat Images.}
\end{figure}

\clearpage
Μετά τις ``καθαρίζουμε'' χρησιμοποιώντας opening στην TOP-Hat, και opening ακολουθούμενο από closing στην BOT-Hat εικόνα.
\begin{figure}[h]
    \centering
    \makebox[\textwidth]{\includegraphics[width=0.6\paperwidth]{6.jpg}}
    \caption{Opened TH and closed BH.}
\end{figure}

\begin{figure}[h]
    \centering
    \makebox[\textwidth]{\includegraphics[width=0.6\paperwidth]{7.jpg}}
    \caption{Final Fused Result.}
\end{figure}

\begin{figure}[h]
    \centering
    \makebox[\textwidth]{\includegraphics[width=\paperwidth]{fin.jpg}}
    \caption{Final Fused Result.}
\end{figure}

\clearpage
Έχουμε θεωρήσει, λοιπόν, αυτές τις παραμέτρους ιδανικές, καθώς έχουμε ένα αξιοπρεπές περίγραμμα της κατοικημένης περιοχής,
στο οποίο μπορούμε να εφαρμόσουμε τεχνικές από το προηγούμενο εργαστήριο και να πάρουμε όλο το χωριό ως μια περιοχή. Και παράλληλα,
έχουμε αναγνωρίσει επιτυχώς, δρόμους και μεμονομένα σπίτια.

\section{Κώδικας}
\matlabscript {main_6}{Η main.}

\end{document}
