\documentclass[11pt]{scrartcl} % Font size

\input{structure.tex} % Include the file specifying the document structure and custom commands
\input{matlab.tex}

\usepackage{fontspec}
\setmainfont{Tinos Nerd Font} %nice font for english and greek

\usepackage{hyperref} %for hyperlinks
\hypersetup{
    colorlinks=true,
    linkcolor=blue,
    filecolor=magenta,
    urlcolor=cyan,
}
%----------------------------------------------------------------------------------------
%	TITLE SECTION
%----------------------------------------------------------------------------------------

\title{
	\normalfont\normalsize
	\textsc{Technical University of Crete, ECE}\\ % Your university, school and/or department name(s)
	\vspace{25pt} % Whitespace
	\rule{\linewidth}{0.5pt}\\ % Thin top horizontal rule
	\vspace{20pt} % Whitespace
	{\Huge Digital Image Processing}\\ % The assignment title

	{\huge Eighth Lab Report}\\ % The assignment title
	\vspace{12pt} % Whitespace
	\rule{\linewidth}{2pt}\\ % Thick bottom horizontal rule
	\vspace{12pt} % Whitespace
}

\author{\LARGE{Τσιαούσης Χρήστος}\\
		\texttt{2016030017}
		\and
		\LARGE{Πρωτοπαπαδάκης Γιώργος}\\
		\texttt{2016030134}}% Your name

\date{\normalsize\today} % Today's date (\today) or a custom date

\begin{document}

\maketitle % Print the title

\section{Σκοπός Εργαστηρίου}

\begin{figure}[h]
    \centering
    \makebox[\textwidth]{\includegraphics[width=0.4\paperwidth]{cameraman.jpg}}
    \caption{Original Image.}
\end{figure}

\textit{Ο Κώδικας βρίσκεται και συγκεντρωτικά στο τελευταίο κεφάλαιο}

\clearpage
Σκοπός του εργαστηρίου είναι η κατανόηση όλων, πλέον, των μεθόδων συνέλιξης
στις δύο διαστάσεις.

\section{Υλοποίηση}
\subsection*{Βήματα 1-2}
Trivial...
\matlabscript{1}{Βήματα 1-2}

\subsection*{Βήματα 3-4}
Το βήμα αυτό υλοποιήθηκε με τις έτοιμες συναρτήσεις του matlab \textit{fft2} και \textit{fftshift}.
Το μόνο ``περίεργο'' από άποψη υλοποίησης είναι ότι κάναμε zero-pad το σήμα κατά 8 pixel, έτσι ώστε
να μπορούμε να το πολλαπλασιάσουμε αργότερα με τον φουριέ του φίλτρου και να έχουμε ίδιες διαστάσεις
με το αποτέλεσμα της συνλελιξης στο πεδίο του χρόνου.

\begin{figure}[h]
    \centering
    \makebox[\textwidth]{\includegraphics[width=0.6\paperwidth]{1.jpg}}
    \caption{$I_{new} and \mathcal{F}\{I_{new}\}$}
\end{figure}

Το σήμα στο πεδίο των συχνοτήτων θυμίζει δύο ορθογώνιους παλμούς raised cosine (ενας επι του xx' κι ένας επι του yy'),

\matlabscript{2}{Βήματα 3-4}

\subsection*{Βήματα 5-6-7}
Για την δημιουργία του φίλτρου, χρησιμοποιήσαμε την \textit{fspecial}, όπως την είδαμε και στο τρίτο εργαστήριο,
και για τον μετασχηματισμό, ακολουθήσαμε την ίδια διαδικασία με το βήμα τρία. Αυτό που παρατηρήσαμε από τα meshes,
αλλά και παίζοντας με την τιμή διασποράς του φίλτρου, είναι ότι ακολουθεί τους κανόνες που ήδη ξέρουμε από τα σήματα
μίας διάστασης. Δηλαδή, όσο πιο στενό το ``καρφί'' στο χωρικό σήμα, τόσο πιο ``απλωμένο'' και κοντά στην μονάδα το συχνοτικό
πέπλο.

\begin{figure}[h]
    \centering
    \makebox[\textwidth]{\includegraphics[width=0.6\paperwidth]{2.jpg}}
    \caption{Gaussian Filter in spatial and in frequency domain.}
\end{figure}

\matlabscript{3}{Βήματα 5-6-7}

\subsection*{Βήματα 8-9-10}

Η συνέλιξη δισδιάστατων σημάτων με την συνάρτηση \textit{conv2}, χωρις επιπλέον ορίσματα, υλοποιείται με την τελευταία
γραμμή του φίλτρου επί την πρώτη του σήματος. Γι αυτό καταλήγουμε με ένα σήμα 38x38. Ευτυχώς, το έχουμε προβλέψει κι έχουμε
τα σήματα της εικόνας και του φίλτρου με ίδιο μέγεθος στις συχνότητες και χωρίς ολίσθηση. Έτσι, για τον πολλαπλασιασμό, αρκεί να
πολλαπλασιάσουμε στοιχείο-στοιχείο και έπειτα να κατασκευάσουμε το σήμα στο πεδίο του χώρου χρησιμοποιώντας την \textit{ifft2}.
\begin{figure}[h]
    \centering
    \makebox[\textwidth]{\includegraphics[width=0.6\paperwidth]{3.jpg}}
    \caption{Traditional convolution and frequency multiplication.}
\end{figure}

\begin{figure}[h]
    \centering
    \makebox[\textwidth]{\includegraphics[width=0.6\paperwidth]{4.jpg}}
    \caption{Convolution using the Frequency Domain}
\end{figure}
\matlabscript{4}{Βήματα 8-9-10}

\subsection*{Βήματα 11-12-13}

Για το βήμα 11, χρησιμοποιήθηκαν μόνο οι συναρτήσεις \textit{zeros(), cell() \& cell2mat()} για convenience στην αποθήκευση
του toeplitz matrix, καθώς και η \textit{circshift()} για convenience στην δημιουργία των πινάκων $H_n$. Κατά τα άλλα υλοποιήσαμε
βήμα-βήμα τις σελίδες 15-23 της ένατης διάλεξης, χωρίς την χρήση άλλων συναρτήσεων όπως toeplitz ή convmtx2, και πλέον μπορούμε
να πούμε με σιγουριά ότι: ναι, όντως η θεωρία στέκει. :ρ


\begin{figure}[h]
    \centering
    \makebox[\textwidth]{\includegraphics[width=0.6\paperwidth]{5.jpg}}
    \caption{Toeplitz Matrix as image and Convoluted Image using Toeplitz Matrix.}
\end{figure}


Ήταν μια ευχάριστη σπαζοκεφαλιά, αλλά σίγουρα η τελευταία φορά που κάνουμε συνέλιξη με αυτή την μεθοδολογία.
Παραθέτουμε το κομμάτι κώδικα και θα συζητήσουμε μετά τα σφάλματα.

\matlabscript{5}{Βήματα 11-12-13}

\section{Συμπεράσματα}

\begin{table}[h] % [h] forces the table to be output where it is defined in the code (it suppresses floating)
    \centering % Centre the table
    \begin{tabular}{l c c c}
        \toprule
        \textit{σφάλμα} & \textbf{(conv2,ifft2)} & \textbf{(conv2,Toeplitz)} & \textbf{(ifft2,Toeplitz)} \\
        \midrule
        \midrule
        MSE & $1.4\cdot 10^{-27}$ & $1.5\cdot 10^{-27}$ & $1.3\cdot 10^{-27}$ \\
        \bottomrule
    \end{tabular}
    \caption{Τα τελικά σφάλματα.}
\end{table}

Φαίνεται ότι τα σφάλματα είναι πάρα πολύ μικρά και αυτό μάλλον οφείλεται σε στρογγυλοποιήσεις που κάνει
το Matlab για floating point αριθμούς κοντά στο μηδέν, ή και στους διαφορετικούς τύπους πινάκων που χρησιμοποιήθηκαν
σε κάθε μέθοδο. Για επιβεβαίωση, αν κάποιος ήταν ``ψείρας'', θα μπορούσε να βεβαιωθεί ότι σε κάθε βήμα η είσοδος
είναι \textit{complex double}, για να έχει κοινούς τύπους παντού. Παρ' όλ' αυτά, έχουμε και ζωή και χίλια δυο μαθήματα.

Σε γενικές γραμμές πάντως δεν θεωρούμε ότι οι εικόνες δεν διαφέρουν καθόλου, καθώς το $10^{-27}$ είναι γελοία μικρό νούμερο.

\section{Κώδικας}

\matlabscript {lab8}{Η main.}



\end{document}
