\documentclass[11pt]{scrartcl} % Font size

%%%%%%%%%%%%%%%%%%%%%%%%%%%%%%%%%%%%%%%%%
% Wenneker Assignment
% Structure Specification File
% Version 2.0 (12/1/2019)
%
% This template originates from:
% http://www.LaTeXTemplates.com
%
% Authors:
% Vel (vel@LaTeXTemplates.com)
% Frits Wenneker
%
% License:
% CC BY-NC-SA 3.0 (http://creativecommons.org/licenses/by-nc-sa/3.0/)
%
%%%%%%%%%%%%%%%%%%%%%%%%%%%%%%%%%%%%%%%%%

%----------------------------------------------------------------------------------------
%	PACKAGES AND OTHER DOCUMENT CONFIGURATIONS
%----------------------------------------------------------------------------------------

\usepackage{amsmath, amsfonts, amsthm} % Math packages

\usepackage{listings} % Code listings, with syntax highlighting

\usepackage[english]{babel} % English language hyphenation

\usepackage[xetex]{graphicx}
%\usepackage{graphicx} % Required for inserting images
\graphicspath{{Figures/}{./}} % Specifies where to look for included images (trailing slash required)

\usepackage{booktabs} % Required for better horizontal rules in tables

\numberwithin{equation}{section} % Number equations within sections (i.e. 1.1, 1.2, 2.1, 2.2 instead of 1, 2, 3, 4)
\numberwithin{figure}{section} % Number figures within sections (i.e. 1.1, 1.2, 2.1, 2.2 instead of 1, 2, 3, 4)
\numberwithin{table}{section} % Number tables within sections (i.e. 1.1, 1.2, 2.1, 2.2 instead of 1, 2, 3, 4)

\setlength\parindent{0pt} % Removes all indentation from paragraphs

\usepackage{enumitem} % Required for list customisation
\setlist{noitemsep} % No spacing between list items

%----------------------------------------------------------------------------------------
%	DOCUMENT MARGINS
%----------------------------------------------------------------------------------------

\usepackage{geometry} % Required for adjusting page dimensions and margins

\geometry{
	paper=a4paper, % Paper size, change to letterpaper for US letter size
	top=2.5cm, % Top margin
	bottom=3cm, % Bottom margin
	left=3cm, % Left margin
	right=3cm, % Right margin
	headheight=0.75cm, % Header height
	footskip=1.5cm, % Space from the bottom margin to the baseline of the footer
	headsep=0.75cm, % Space from the top margin to the baseline of the header
	%showframe, % Uncomment to show how the type block is set on the page
}

%----------------------------------------------------------------------------------------
%	FONTS
%----------------------------------------------------------------------------------------

\usepackage[utf8]{inputenc} % Required for inputting international characters
\usepackage[T1]{fontenc} % Use 8-bit encoding

\usepackage{fourier} % Use the Adobe Utopia font for the document

%----------------------------------------------------------------------------------------
%	SECTION TITLES
%----------------------------------------------------------------------------------------

\usepackage{sectsty} % Allows customising section commands

\sectionfont{\vspace{6pt}\centering\normalfont\scshape} % \section{} styling
\subsectionfont{\normalfont\bfseries} % \subsection{} styling
\subsubsectionfont{\normalfont\itshape} % \subsubsection{} styling
\paragraphfont{\normalfont\scshape} % \paragraph{} styling

%----------------------------------------------------------------------------------------
%	HEADERS AND FOOTERS
%----------------------------------------------------------------------------------------

\usepackage{scrlayer-scrpage} % Required for customising headers and footers

\ohead*{} % Right header
\ihead*{} % Left header
\chead*{} % Centre header

\ofoot*{} % Right footer
\ifoot*{} % Left footer
\cfoot*{\pagemark} % Centre footer
 % Include the file specifying the document structure and custom commands
% LaTeX settings for MATLAB code listings
% based on Ted Pavlic's settings in http://links.tedpavlic.com/ascii/homework_new_tex.ascii
\usepackage{listings}
\usepackage[usenames,dvipsnames]{color}

% This is the color used for MATLAB comments below
\definecolor{MyDarkGreen}{rgb}{0.0,0.4,0.0}

% For faster processing, load Matlab syntax for listings
\lstloadlanguages{Matlab}%
\lstset{language=Matlab,                        % Use MATLAB
        frame=single,                           % Single frame around code
        basicstyle=\scriptsize\ttfamily,             % Use small true type font
        keywordstyle=[1]\color{Blue}\bfseries,        % MATLAB functions bold and blue
        keywordstyle=[2]\color{Purple},         % MATLAB function arguments purple
        keywordstyle=[3]\color{Blue}\underbar,  % User functions underlined and blue
        identifierstyle=,                       % Nothing special about identifiers
                                                % Comments small dark green courier
        commentstyle=\usefont{T1}{pcr}{m}{sl}\color{MyDarkGreen}\small,
        stringstyle=\color{Purple},             % Strings are purple
        showstringspaces=false,                 % Don't put marks in string spaces
        tabsize=3,                              % 5 spaces per tab
        %
        %%% Put standard MATLAB functions not included in the default
        %%% language here
        morekeywords={xlim,ylim,var,alpha,factorial,poissrnd,normpdf,normcdf,imresize,double,immse,fspecial,cell2mat,circshift,cell},
        %
        %%% Put MATLAB function parameters here
        morekeywords=[2]{on, off, interp},
        %
        %%% Put user defined functions here
        morekeywords=[3]{FindESS, homework_example},
        %
        morecomment=[l][\color{Blue}]{...},     % Line continuation (...) like blue comment
        numbers=left,                           % Line numbers on left
        firstnumber=1,                          % Line numbers start with line 1
        numberstyle=\tiny\color{Blue},          % Line numbers are blue
        stepnumber=1                            % Line numbers go in steps of 5
        }

% Includes a MATLAB script.
% The first parameter is the label, which also is the name of the script
%   without the .m.
% The second parameter is the optional caption.
\newcommand{\matlabscript}[2]
  {\begin{itemize}\item[]\lstinputlisting[caption=#2,label=#1]{#1.m}\end{itemize}}


\usepackage{fontspec}
\setmainfont{Tinos Nerd Font} %nice font for english and greek

\usepackage{hyperref} %for hyperlinks
\hypersetup{
    colorlinks=true,
    linkcolor=blue,
    filecolor=magenta,
    urlcolor=cyan,
}
%----------------------------------------------------------------------------------------
%	TITLE SECTION
%----------------------------------------------------------------------------------------

\title{
	\normalfont\normalsize
	\textsc{Technical University of Crete, ECE}\\ % Your university, school and/or department name(s)
	\vspace{25pt} % Whitespace
	\rule{\linewidth}{0.5pt}\\ % Thin top horizontal rule
	\vspace{20pt} % Whitespace
	{\Huge Digital Image Processing}\\ % The assignment title

	{\huge Project Report}\\ % The assignment title
	\vspace{12pt} % Whitespace
	\rule{\linewidth}{2pt}\\ % Thick bottom horizontal rule
	\vspace{12pt} % Whitespace
}

\author{\LARGE{Τσιαούσης Χρήστος}\\
		\texttt{2016030017}
		\and
		\LARGE{Πρωτοπαπαδάκης Γιώργος}\\
		\texttt{2016030134}}% Your name

\date{\normalsize\today} % Today's date (\today) or a custom date

\begin{document}

\maketitle % Print the title

\section{Σκοπός}

\begin{figure}[h]
    \centering
    \makebox[\textwidth]{\includegraphics[width=0.9\paperwidth]{pipeline.jpg}}
    \caption{T/R pipeline.}
    \label{fig:1}
\end{figure}

\textit{*Ο Κώδικας βρίσκεται και συγκεντρωτικά στο τελευταίο κεφάλαιο}

\clearpage
Σκοπός του project ήταν η κατανόηση του παραπάνω pipeline, του όγκου πληροφορίας που κρύβεται στα multimedia και η υλοποίηση βασικών τεχνικών κβάντισης και κωδικοποίησης
έτσι ώστε να ελαχιστοποιήσουμε κατά πολύ την μεταδιδόμενη πληροφορία.

\section{Μέρος Πρώτο}

\subsection{Σε μονοδιάστατα σήματα}
\begin{figure}[h]
    \centering
    \makebox[\textwidth]{\includegraphics[width=0.8\paperwidth]{1.jpg}}
    \caption{Η Χαρακτηριστική συνάρτηση του κβαντιστή.}
    \label{quant1d}
\end{figure}

Όπως φαίνεται και από την χαρακτηριστική συνάρτηση \ref{quant1d}, υλοποιήσαμε έναν \textbf{mid-tread quantizer} και του δώσαμε ένα πολύ πυκνό (με βήμα 0.01) σήμα
εισόδου, έτσι ώστε να φανεί η περιοχή deadzone. Η υλοποίηση ήταν απλή εφαρμογή του τύπου που δίνεται στην εκφώνηση και βρίσκεται \hyperlink{uni_scalar}{εδώ}.


\subsection{Σε δισδιάστατα σήματα}
\begin{figure}[h]
    \centering
    \makebox[\textwidth]{\includegraphics[width=0.8\paperwidth]{2.jpg}}
    \caption{Η αγαπημένη Λένα κβαντισμένη σε κάθε configuration.}
    \label{quant2d}
\end{figure}

Για την επιτυχή κβάντιση της εικόνας, υλοποιήσαμε την περίπτωση του δισδιάστατου σήματος στην \textbf{uni\_scalar} και κάναμε την εξής παρατήρηση:

Αφού η εικόνα μας είναι grayscale και 8bit αντιστοιχούν για κάθε pixel, αυτό σημαίνει ότι το άνω όριο στις τιμές των pixel είναι $2^8-1$.
Αυτό σημαίνει ότι θα πρέπει να ``μεταφραστεί'' όπως φαίνεται παρακάτω.
\[
A\in [-127, 127] \Rightarrow
\boxed{A\in [0, 255]}
\]

Γι αυτό τον λόγω λοιπόν, καλούμε την uni\_scalar με όρισμα Α/2 και όχι Α. Τέλος μπορούμε να δούμε οπτικά τα επίπεδα κβάντισης να αυξάνονται, για παράδειγμα,
η πρώτη εικόνα στο \ref{quant2d} έχει ένα απόλυτο επίπεδο, που σημαίνει ένα στα θετικά κι ένα στα αρνητικά. Αυτό έπειτα μεταφράζεται βάσει της παραπάνω σχέσης
και δίνει τις τιμές που μας επιβεβαιώνει και το \textit{colorbar:} 0 και 255.

\subsection{Σφάλματα}
\begin{figure}[h]
    \centering
    \makebox[\textwidth]{\includegraphics[width=0.8\paperwidth]{3.jpg}}
    \caption{Το μέσο τετραγωνικό σφάλμα, ανά επίπεδο κβάντισης.}
    \label{mse}
\end{figure}
\clearpage

Το διάγραμμα των σφαλμάτων \ref{mse} είναι σε λογαριθμική κλίμακα. Με αυτό τον τρόπο βλέπουμε πολύ πιο εύκολα τις μεγάλες διακυμάνσεις στις τιμές. Ακόμα και για
βήμα κβάντισης ίσο με την μονάδα \textbf{$$Q_{step}=1 \Leftrightarrow (A=127, L=2^8),$$} όπου έχουμε προσεγγίσει κατά πολύ το 0, αλλά δεν το φτάνουμε ποτέ. Αυτό κατά πάσα
πιθανότητα ωφείλεται στην \textbf{περιοχή deadzone}.

Η υλοποίηση του πρώτου μέρους βρίσκεται \hyperlink{partA}{εδώ}.

\section{Μέρος Δεύτερο}

Το βίντεο έχει \textbf{141 frames}, με ρυθμό \textbf{30 fps}, η ανάλυση του κάθε ενός είναι \textbf{320x240} και τέλος η διάρκεια \textbf{4.7 seconds}. Οι αριθμοί επιβεβαιώνονται
μεταξύ τους και όποιο frame και να ``πάρουμε'', έχει πάντα την ίδια ανάλυση. Από άποψη υλοποίησης, δεν χρησιμοποιήθηκε η \textit{imshow}, γιατι δεν έκανε σωστό allign στους άξονες
του figure.
\begin{figure}[h]
    \centering
    \makebox[\textwidth]{\includegraphics[width=0.8\paperwidth]{4.jpg}}
    \caption{Frame number 50.}
    \label{50}
\end{figure}

 Αυτό που έχει μεγάλο ενδιαφέρον ως προς τον σχολιασμό, είναι ότι αφού κάναμε την αναπαραγωγή του video, έπρεπε να γίνει reinitialize
το αντικείμενο για να πάμε σε συγκεκριμένο frame. Η πρώτη σκέψη είναι ``Μα καλά δεν μπορούσαν να το υλοποιήσουν;'', αλλά οι αστοχίες που υπήρχαν με την κλάση \textit{VideoReader},
μας έδωσαν hints για όλο το backend που χρησιμοποιείται στην αναπαραγωγή βιντεο. Οι βιβλιοθήκες ffmpeg και GStreamer είναι ο ελβετικός σουγιάς των multimedia. Παρέχουν API για πολλές
γλώσσες προγραμματισμού καθώς και εφαρμογές για terminal. Οι δυνατότητες είναι ατελείωτες. Με την απλή εντολή \lstinline{ffmpeg -i xylophone3.mp4} μπορούμε να δούμε αναλυτικά τα metadata
του αρχείου και πως έχει video encoding \textbf{h264}.


\begin{figure}[h]
    \centering
    \makebox[\textwidth]{\includegraphics[width=0.6\paperwidth]{h264.jpg}}
    \caption{h264 encoder/decoder pipeline.}
    \label{h264}
\end{figure}


Πολύ συχνά στα βιντεο υπάρχει ένα \textbf{σταθερό background} και ένα πολύ
μικρό ποσοστό των pixel αλλάζουν από frame σε frame. Αυτό ακριβώς εκμεταλεύεται αυτή η κωδικοποίηση και δεν μπορείς να μάθεις την πληροφορία ολόκληρου του frame χωρίς να ξέρεις το προηγούμενο.
\textbf{Γι αυτό πρέπει να γίνει reinitialize το αντικείμενο!}. Τέλος στο \ref{h264} φαίνεται το pipeline της κωδικοποίησης αυτής, που προς θετική μας έκπληξη, απέχει ελάχιστα από το ``βασικό''
που κάνουμε σε αυτό το project.

Για να εκτιμηθεί η αξία του, ενδεικτικά το βίντεο που είχαμε να επεξεργαστούμε σε raw μορφή θα έπιανε περίπου 87 εκατομμύρια bit, ενώ τώρα είναι 1,2 εκατομμύρια bit!!

Δείτε τον κώδικα αυτού του μέρους \hyperlink{partB}{εδώ}.

\clearpage
\section{Μέρος Τρίτο}

\begin{figure}[h]
    \centering
    \makebox[\textwidth]{\includegraphics[width=0.4\paperwidth]{haar.jpg}}
    \caption{2 level HAAR wavelet tansform for 2-D signal.}
    \label{haar}
\end{figure}


Σε αυτό το μέρος συνδυάσαμε τον κβαντιστή από το πρώτο μέρος και εξελήξαμε την μέθοδο Haar, που \hyperlink{haar1}{έχει υλοποιηθέι} σε προηγούμενο εργαστήριο, ώστε να λειτουργεί με 2-D σήματα.
Ο μετασχηματισμός αυτός χωρίζει την εικόνα
σε τέσσερα τεταρτημόρια ανά επίπεδο. Κάθε επίπεδο παίρνει το άνω αριστερά του προηγουμένου για να κάνει την επεξεργασία, όπως έχουμε δει και στην θεωρία.
Κάναμε έτσι μετασχηματισμό δυο επιπέδων στο grayscaled πεντηκοστό frame του `xylophone3.mp4' όπως φαίνεται στο \ref{c1}.

\begin{figure}[h]
    \centering
    \makebox[\textwidth]{\includegraphics[width=0.8\paperwidth]{5.jpg}}
    \caption{2 level HAAR wavelet tansform for 2-D signal.}
    \label{c1}
\end{figure}

(Υπενθύμιση: όλος ο κώδικας στο τελευταίο κεφάλαιο)

\subsection{Αποφάσεις πριν την Κβάντιση}
Όπως έχει επισημανθεί και επάνω στο \ref{c1}, μπορεί κανείς να παρατηρήσει ότι οι τιμές στα τεταρτημόρια $H6 \rightarrow H1$ κυμαίνονται από -50 εώς και 255 (στα σημεία που είναι λευκά).
Κι έτσι πρέπει να δώσουμε μεγαλύτερο παράθυρο κβάντισης από το Α/2 που είδαμε στο πρώτο μέρος. Αν και το ένστικτο λέει ότι χοντρικά το Α για την κβάντιση των subbands θα πρέπει να είναι

\lstinline{abs( min(min(subband)) + max(max(subband)) ) / 2},

πειραματικά είδαμε πως το καλύτερο σφάλμα σηματοθορυβικής σχέσης συγκριτικά με την αρχική εικόνα το πήραμε για Α ίσο με την
ελάχιστη τιμή pixel ολόκληρης της εικόνας και ίσως αν δεν δίναμε και μαθήματα σε πέντε μέρες να βρίσκαμε και βέλτιστη τιμή για το επόμενο βήμα (που έχουμε διαφορετικά επίπεδα κβάντισης).
Τέλος να πούμε ότι ελέγχεται η σωστή σύνθεση haar με μέσο τετραγωνικό σφάλμα ( που βγαίνει 0 ).

\subsection{Εντροπία}
Έπειτα υπολογίσαμε την εντροπία του κάθε subband ξεχωριστά, αλλά και ολόκληρης της εικόνας με την βοήθεια των συναρτήσεων που υλοποιήσαμε. Η συνολική εντροπία έπειτα από την κωδικοποίηση
και την κβάντιση έπεσε στό \textbf{1.257}(!) κι έτσι βρήκαμε λόγο εντροπιών αρχικής και τελικής εικόνας \textbf{6.02}.

\begin{figure}[h]
    \centering
    \makebox[\textwidth]{\includegraphics[width=0.4\paperwidth]{matl.jpg}}
    \caption{Εντροπία από την κονσόλα του Matlab.}
    \label{ent}
\end{figure}

[Spoiler Alert] παραθέτουμε και του output της κονσόλας του matlab για τις τιμές της εντροπίας του κάθε subband στην \ref{ent}.

\subsection{Ψάχνωντας τα bits}
Θέλαμε όμως να βρούμε και την πραγματική συμπίεση. Ψάχνοντας λίγο
είδαμε ότι η συνάρτηση \textit{whos()} δίνει πληροφορίες για οποιαδήποτε κλάση στο matlab, αλλά δεν μας έκανε επειδή δεν μπορέσαμε να κάνουμε μεταβλητά τα bit για κάθε pixel (no memory padding).
Έτσι τα υπολογίσαμε ``χειροκίνητα'' βρίσκοντας τα bit της αρχικής εικόνας με την σχέση $m\cdot n\cdot 8$ και τα bit της τελικής με την σχέση
$$(\frac{m}{4}\cdot \frac{n}{4}\cdot 8) + 3(\frac{m}{4}\cdot \frac{n}{4}\cdot QBits_{lvl2}) + 3(\frac{m}{2}\cdot \frac{n}{2}\cdot QBits_{lvl1}),$$
όπου ο πρώτος όρος αντιστοιχεί στο Η7, ο δεύτερος στα Η6, Η5, Η4, ο τρίτος στα Η3, Η2, Η1 και τα QBits καμία σχέση με την κβαντομηχανική.

Έτσι, καταλήξαμε σε \textbf{πραγματικο ρυθμό συμπίεσης 1.88} για την πρώτη περίπτωση. Πράγμα το οποίο σημαίνει πως αν μας ενδιέφερε η αποστολή της εικόνας σε ένα τηλεπικοινωνιακό κανάλι
θα είχαμε μεγαλύτερες ταχύτητες καθαρά και μόνο από τα bit κωδικοποίησης αλλά και μεγαλύτερη ανοχή στον θόρυβο λόγο κβάντισης. Η εικόνα που παρήγαγε η μεθοδολογία μας είναι η \ref{c2}.

\begin{figure}[h]
    \centering
    \makebox[\textwidth]{\includegraphics[width=0.8\paperwidth]{6.jpg}}
    \caption{[part C] method 1.}
    \label{c2}
\end{figure}

Τελείως αντίστοιχα, η δεύτερη μέθοδος όπου κβαντίζει με 3bit το πρώτο επίπεδο και με 5 το δεύτερο έδωσε ελαφρώς διαφορετικό \hyperlink{apotel}{αποτελέσμα}. Δηλαδή
\textbf{6.09 συμπίεση εντροπίας και 2.17 συμπίεση bit}. Αυτό συνέβει επειδή διαλέξαμε λιγότερα bit αναπαράστασης της low-pass πληροφορίας και σταδιακά όλο και
μεγαλύτερα όσο πηγαίναμε προς την εικόνα χαμηλής ανάλυσης (Η7).

\hypertarget{apotel}{
\begin{figure}[h]
    \centering
    \makebox[\textwidth]{\includegraphics[width=0.8\paperwidth]{7.jpg}}
    \caption{[part C] method 2.}
    \label{c3}
\end{figure}
}

Τέλος βλέπουμε ότι η σηματοθορυβική σχέση είναι αρνητική. Αυτο ωφείλεται στο ότι για πολλά pixel οι τιμές διαφέρουν ελαφρώς και ο λόγος σήματος προς τον θόρυβο είναι μικρότερος της μονάδας, άρα
και ο λογάριθμος αρνητικός. Παρ`όλ'αυτά δεν τα πήγαμε και τόσο άσχημα, ειδικά αναλογιζόμενοι πως δεν κάναμε de-Quantization. Και πάνω σε αυτό να πούμε ότι αν και η δεύτερη μέθοδος έχει χειρότερο
psnr metric, στο μάτι φαίνεται αρκετά καλύτερη και λιγότερο ``pixelιασμένη''. Αυτό μας επιβεβαιώνει ότι οι μετρήσεις σφαλμάτων είναι μόνο ένας δείκτης κι όχι τι αντιλαμβάνεται το μάτι.


Κλείνοντας να πούμε ότι το project ήταν πολύ ενδιαφέρον, όπως και όλα τα υπόλοιπα εργαστήρια του μαθήματος, και σίγουρα μας έκαναν να βλέπουμε την επεξεργασία εικόνας με μια διαφορετική ματιά.
\clearpage
\section{Κώδικας}

\hfill \textit{Οι υπογραμμισμένες συναρτήσεις είναι αυτές που έχουν υλοποιηθεί στα πλαίσια του project.} \hfill

\matlabscript {main}{Η main.}

\hypertarget{uni_scalar}{
    \matlabscript {uni_scalar}{Η υλοποίηση του κβαντιστή.}
}

\matlabscript {haar2anal}{Η υλοποίηση της δισδιάστατης ανάλυσης Haar.}

\matlabscript {haar2syn}{Η υλοποίηση της δισδιάστατης σύνθεσης Haar.}

\hypertarget{haar1}{
    \matlabscript {haar1analysis}{Η υλοποίηση της μονοδιάστατης ανάλυσης Haar.}
}

\matlabscript {haar1synthesis}{Η υλοποίηση της μονοδιάστατης σύνθεσης Haar.}

\matlabscript {imEntropy}{Η υλοποίηση υπολογισμού εντροπίας κατά Shannon.}

\matlabscript {entrOfSubBands}{Υπολογισμός εντροπίας για κάθε sub band ξεχωριστά.}

\hypertarget{partA}{
    \matlabscript {partA}{Η υλοποίηση του πρώτου μέρους.}
}

\hypertarget{partB}{
    \matlabscript {partB}{Η υλοποίηση του δεύτερου μέρους.}
}

\hypertarget{partC}{
    \matlabscript {partC}{Η υλοποίηση του τρίτου μέρους.}
}

\end{document}
